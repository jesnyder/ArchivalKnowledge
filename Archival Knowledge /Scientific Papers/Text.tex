\documentclass[a4paper, 11pt]{article}
\usepackage[left=3cm, right=3cm, top=2cm]{geometry}

\usepackage[numbers]{natbib}
\bibliographystyle{unsrtnat}

\setcounter{tocdepth}{1}  % Set the table of contents depth to one header

\begin{document}

\author{Jessica E. Snyder jessica.e.snyder@nasa.gov}
\title{ Research}
\maketitle

\tableofcontents
%\newpage


\section{3D Printing Microfluidic}


Abstract begin 
In the past few years, 3D printing technology has witnessed an explosive growth, penetrating various aspects of our lives. Current best-in-class 3D printers can fabricate micrometer scale objects, which has made fabrication of microfluidics  devices  possible.The  smallest  achievable  resolution  is already  at  nanometer  scale,  which  is continuing to drop. Since geometric complexity is not a concern for 3D printing, novel 3D microfluidics and lab-on-a-chip systems that are otherwise impossible to produce with traditional 2D microfabrication technology have started to emerge in recent years. In this review, we first introduce the basics of3D printing technology for the microfluidic  community and then summarize  its  emerging  applications  in creating  novel microfluidic  devices(MFDs).We foresee  wide spread utilization of 3D printing for future developments in microfluidic engineering and lab-on-a-chip technology.
   \cite{yazdi20163d}
   Abstract over 
   
   Hello. This is how you cite. \cite{gustafsson2016best}
   
   Another reference \cite{hamid2015surface}
   
    \section{The Effect of Space}
    \subsection{NASA's Twin Study}
    
    \subsubsection{Motivating Healthy Behavior}
    Data collected from wearable sensors establish a baseline normal to compare one individual's health to objective standards as part of a health exam. Over time, these measurement establish a normative reference from which abnormalities can be flagged for further scrutiny as possible early signs of disease. The quantifiable feedback motivates patients to take a more active role in their health by providing another window into the cause and effect relationship between lifestyle and health. 
    Self monitoring individuals control their expressive behavior to maintain social appropriateness by observing the other people's reactions  \cite{snyder1974self}. Non-self monitoring individuals disregard situational interpersonal specifications. Instead, they rely on internal emotional state. A Self-Monitoring scale measured the ability to intentionally communicate emotion, actors scored higher than university students who scored higher than psychiatric ward patients.
    
    Frequent and continuous monitoring by wearable biosensors detect disease earlier - shown for Lyme disease and inflammatory response \cite{li2017digital}. Longitudinal sampling of an individual's physiology provides an activity-based normalized baseline to then identify abnormal signals. Sensors help people manage their health. Sensors also expose physiology's environmental sensitivity.  Airline flights decrease peripheral capillary oxygen saturation and increase radiation exposure causing macro-phenotype fatigue. 
    Data-driven feedback motivates individuals to sustained behavioral changes in favor of healthy environmental patterns based on their personal genomic and physiology characteristics \cite{kellogg2018personal}. 
    
    
     Mike Snyder wore sensors - a different post doc for each sensor. Exposome study - what fungus are people exposed to? Heart rate and skin temperature to detect lyme disease - Snyder. 
     
    
     Identical twins Scott and Mark Kelly 
     
    (October 15, 2018) The NASA Twin Study: A multidimensional analysis of a yearlong human spaceflight. Presenter Dr. Tejaswin Mishra - Postdoctoral Fellow in Mike Snyder Lab of  Department of Genetics Stanford University. 
    Humans have been going to space since the early 1957 - 536 people flown to space. What's next?
    The future is Mars. Current travel time hours (will be days from Deep Space Gateway and months from Mars.) Space hazards include muscle atrophy, plasma volume contraction, mood, radiation, access to immediate medical care. Risk categories include mitigated and no-go. 
    Human studies needed to identify risks and knowledge gaps. Plenty of opportunity, but low N. Longitudinal multi-omics deep profiling of one person to predict diabetes, effect of lifestyle changes.
    
    NASA One Year Mission Project - sampling before, during and after for both twins. 
    
    Hypoxia could exist even if oxygen and carbon dioxide in cabin is normal - could be local hypoxia. 
    Results: Gene expression changes during spaceflight - some up and some down, some persist some revert (hypoxia and immune). Methylation levels and methalytion entropy - no substantional changes genome wide - some localized changes - immune signaling, environmental adaptation. 
    Cytokine signaling. - immune response and inflammation during spaceflight (Il-10 signaling)   - 5 days after return a spike of cytokines. Mark reported skid rash after return. 
    Pattern of complex lipids (metabolomics) Complex lipids containing fatty acids that are metabolized to form pro and anti-inflammation molecules are present at different levels in the twin and increase inflight 
    AA is aa precursor of inflammatory molecules - complex lipids containing fatty acids that are metabolized to form pro- and anti-inflammation
    Microbiome (stoll shotgun metagenome) twins same? no Flight associated loss of diversity? no Come back to "normal": somewhat 
    F/B ratio, Shannon diversity and community structure: changes due to isolation and diet change consistent with ground studies. 
    Products of microbial metabolism 
    Mitochondria-related adaptations during flight (sea horse assay) get oxygen consumption rate. Isolate elements of respiration. Maximal - Basal = Spare reserve capacity to find ATP linked respiration  to compare flight to ground.  Maximal unchanged. Lower reserve can be stress, ischemia, cell death, heart disease, body mass changes (Scott exercising, lost 7\% of body mass). 
    Inflight shift towards anaerobic metabolism - similar results in plasma (could be due to body mass changes) not much evidence for hypoxia based on O2 and CO2 on station. 
    Cardiovascular changes during spaceflight - artery wall thickens, diameter shrinks. Profibrotic markers effect cardiac fibrosis - increase associated with vascular wall dimensions
    Body mass and nutrition changes - lost urine volume and plasma volume is a dehydration response due to loow water intake and low humidity
    Body mass and nutrition changes - bone go up early in flight, down late in flight. tracks with exercise
    Summary - low risk - transcriptional and metabolic changes, teleomere changes, microbiome. 
    Neuro-ocular changes - fluid shift in space flattens colloidal folds - space increased Protein LRG1 down regulated in space flight 
    Telomere changes - DNA damage response inflight. Telomere lengthens in space, retracts when back to ground. Distributed differently across different cell populations. Critically short telomere length. Chromosonal abbroations - inversions are higher. Martha for telomere elongation observed - maybe Susan Daly. In cancer cells different kind of lengthening to help cancer cells proliferate and divide. Longer telomere  can lead to cancer?
    Change in hydration levels? Mt Everest study as analog for extreme environments. 
    
    Accumulated stressors cause physiological damage, which manifests a cumulative disadvantage for an individual's health prospects. Shortened telomeres has been coorelated with increased cancer risk  \cite{latham2017exploring}. Some evidence suggests shortened women with more stress presented shortened telomere length. 
    Short telomere may predict breast cancer for patients who also a BRCA2 mutation \cite{thorvaldsdottir2017telomere}. In part, BRCA2 protects telomere length. When BRCA2 mutates, the risk of breast cancer increases. Women with breast cancer have significantly shorter telomere - as measured using blood cells analyzed by a high throughput monochrome multiplex qPCR method. Having or not having the BRCA2 mutation did not effect teleomere length after a breast cancer diagnosis. Before diagnosis, patients carrying a BRCA2 mutation had shorter telomere lengths than non-carriers. No coorelation was found with survival.   Depression severity caused no differences in leukocyte telomere length, neither did insulin-resistant and insulin-sensitive subject \cite{rasgon2017insulin}. Patient with longer telomeres experience greater relief of depression severity.
    
   
   
   \section{Sponge}
Animals and sponge have a common ancestor 600 million ago: One that gave both branches of life the genes responsible for multi-cellularity (cell adhesion, metazoan body plans), a recognition of self (allorecognition), and recognition of non-self (innate immunity). Normative physiology relies on these genes to solicit coordinated function from a 3-dimensional collection of individual cells. Cancerous growth can arise when these genes responsible for basic cell processes become defective \cite{srivastava2010amphimedon}. These studies were conducted using the Amphimedon queenslandica (demosponge from Australia's Great Barrier Reef) \cite{srivastava2010amphimedon}. 
Scientists observed only a fraction of sponge species regenerate functional tissues from dissociated cells. Most sponge species disintegrate before progressing through all four regeneration checkpoints. Two sponge species - Spongilla lacustris and Haliclona cf. permollis - have adapted to regenerate anatomy comprised of specialized cells with a coordinated function \cite{eerkes2015sponge}. 
Sponges feed by heterotrophy, relying on their environment for nutrients, collected by filtering water and hosting a symbiot of bacteria. Water passes through a manifold of interconnected choanocyte chambers lined with specialized cells which capture nutrients - a seemingly metozoan departure from the free-living protozoa they host.  Sponge's anatomical architecture throttles the water flow rate top optimize digestion \cite{riesgo2014analysis}.
If ancestral sponge ever had anatomical complexity beyond choanocyte chambers - it has been lost or concealed. 
 
Transcriptome sequences reflect environmental conditions - they also expose phylogenic relationships. 

Sponges (Porifera) are among the earliest evolving metazoans. Their filter-feeding body plan based on choanocyte chambers organized into a complex aquiferous system is so unique among metazoans that it either reflects an early divergence from other animals prior to the evolution of features such as muscles and nerves, or that sponges lost these characters. Analyses of the Amphimedon and Oscarella genomes support this view of uniqueness?many key metazoan genes are absent in these sponges?but whether this is generally true of other sponges remains unknown. We studied the transcriptomes of eight sponge species in four classes (Hexactinellida, Demospongiae, Homoscleromorpha, and Calcarea) specifically seeking genes and pathways considered to be involved in animal complexity. For reference, we also sought these genes in transcriptomes and genomes of three unicellular opisthokonts, two sponges (A. queenslandica and O. carmela), and two bilaterian taxa. Our analyses showed that all sponge classes share an unexpectedly large complement of genes with other metazoans. Interestingly, hexactinellid, calcareous, and homoscleromorph sponges share more genes with bilaterians than with nonbilaterian metazoans. We were surprised to find representatives of most molecules involved in cell?cell communication, signaling, complex epithelia, immune recognition, and germ-lineage/sex, with only a few, but potentially key, absences. A noteworthy finding was that some important genes were absent from all demosponges (transcriptomes and the Amphimedon genome), which might reflect divergence from main-stem lineages including hexactinellids, calcareous sponges, and homoscleromorphs. Our results suggest that genetic complexity arose early in evolution as shown by the presence of these genes in most of the animal lineages, which suggests sponges either possess cryptic physiological and morphological complexity and/or have lost ancestral cell types or physiological processes.

   \section{Animal Generated Seismic Activity}
The Earth vibrates with information. Wild elephants produce ground-based vibrations during running, vocalization, and stampeding - as caused by poaching threats. Classifying seismic signatures as specific elephant behavior requires deconstruction of geophone measurements - parsing natural vs. anthropoenic, then subtraction of time-varying seismic activity, and accounting for the terrain \cite{mortimer2018classifying}. The elephant behavior produced a seismic wave. The wave contained information about it was created - time, place, intensity, number of elephants, movements, speed.   
Seismic vibrations generated by wild elephants were recorded in Kenya (Supplemental Information). We selected a few examples of each observed behaviour type, as well as car noise, which were processed to determine the corresponding source function ? the force strength and pattern generated by the elephant ?at the source? (Supplemental Information). Differences in elephant behaviour caused detectable changes in source function properties, which remained distinguishable during modelled seismic wave propagation up to 1000 metres regardless of the noise level and terrain type (Figure 1; Supplemental Information). Recordings of seismic vibrations can therefore be used to classify elephant behaviours.



     \bibliographystyle{plainnat} 
    \bibliography{References}
\end{document}






