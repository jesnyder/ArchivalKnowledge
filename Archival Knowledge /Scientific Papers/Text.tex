\iffalse 
OS X command script to compile LaTex:
	pdflatex Text.tex
	latexmk -pdf Text.tex
	latexmk -pdf -bibtex Text.tex
\fi

\documentclass{article}
\usepackage[numbers]{natbib}

\begin{document}

	
   Title
   
   Hello. This is how you cite. \cite{gustafsson2016best}
   
   Another reference \cite{hamid2015surface}
   
   \section{Sponge}
Animals and sponge have a common ancestor 600 million ago: One that gave both branches of life the genes responsible for multi-cellularity (cell adhesion, metazoan body plans), a recognition of self (allorecognition), and recognition of non-self (innate immunity). Normative physiology relies on these genes to solicit coordinated function from a 3-dimensional collection of individual cells. Cancerous growth can arise when these genes responsible for basic cell processes become defective \cite{srivastava2010amphimedon}. These studies were conducted using the Amphimedon queenslandica (demosponge from Australia's Great Barrier Reef) \cite{srivastava2010amphimedon}. 
Scientists observed only a fraction of sponge species regenerate functional tissues from dissociated cells. Most sponge species disintegrate before progressing through all four regeneration checkpoints. Two sponge species - Spongilla lacustris and Haliclona cf. permollis - have adapted to regenerate anatomy comprised of specialized cells with a coordinated function \cite{eerkes2015sponge}. 
Sponges feed by heterotrophy, relying on their environment for nutrients, collected by filtering water and hosting a symbiot of bacteria. Water passes through a manifold of interconnected choanocyte chambers lined with specialized cells which capture nutrients - a seemingly metozoan departure from the free-living protozoa they host.  Sponge's anatomical architecture throttles the water flow rate top optimize digestion \cite{riesgo2014analysis}.
If ancestral sponge ever had anatomical complexity beyond choanocyte chambers - it has been lost or concealed. 
 
Transcriptome sequences reflect environmental conditions - they also expose phylogenic relationships. 

Sponges (Porifera) are among the earliest evolving metazoans. Their filter-feeding body plan based on choanocyte chambers organized into a complex aquiferous system is so unique among metazoans that it either reflects an early divergence from other animals prior to the evolution of features such as muscles and nerves, or that sponges lost these characters. Analyses of the Amphimedon and Oscarella genomes support this view of uniqueness?many key metazoan genes are absent in these sponges?but whether this is generally true of other sponges remains unknown. We studied the transcriptomes of eight sponge species in four classes (Hexactinellida, Demospongiae, Homoscleromorpha, and Calcarea) specifically seeking genes and pathways considered to be involved in animal complexity. For reference, we also sought these genes in transcriptomes and genomes of three unicellular opisthokonts, two sponges (A. queenslandica and O. carmela), and two bilaterian taxa. Our analyses showed that all sponge classes share an unexpectedly large complement of genes with other metazoans. Interestingly, hexactinellid, calcareous, and homoscleromorph sponges share more genes with bilaterians than with nonbilaterian metazoans. We were surprised to find representatives of most molecules involved in cell?cell communication, signaling, complex epithelia, immune recognition, and germ-lineage/sex, with only a few, but potentially key, absences. A noteworthy finding was that some important genes were absent from all demosponges (transcriptomes and the Amphimedon genome), which might reflect divergence from main-stem lineages including hexactinellids, calcareous sponges, and homoscleromorphs. Our results suggest that genetic complexity arose early in evolution as shown by the presence of these genes in most of the animal lineages, which suggests sponges either possess cryptic physiological and morphological complexity and/or have lost ancestral cell types or physiological processes.

   \section{Animal Generated Seismic Activity}
The Earth vibrates with information. Wild elephants produce ground-based vibrations during running, vocalization, and stampeding - as caused by poaching threats. Classifying seismic signatures as specific elephant behavior requires deconstruction of geophone measurements - parsing natural vs. anthropoenic, then subtraction of time-varying seismic activity, and accounting for the terrain \cite{mortimer2018classifying}. The elephant behavior produced a seismic wave. The wave contained information about it was created - time, place, intensity, number of elephants, movements, speed.   
Seismic vibrations generated by wild elephants were recorded in Kenya (Supplemental Information). We selected a few examples of each observed behaviour type, as well as car noise, which were processed to determine the corresponding source function ? the force strength and pattern generated by the elephant ?at the source? (Supplemental Information). Differences in elephant behaviour caused detectable changes in source function properties, which remained distinguishable during modelled seismic wave propagation up to 1000 metres regardless of the noise level and terrain type (Figure 1; Supplemental Information). Recordings of seismic vibrations can therefore be used to classify elephant behaviours.



     \bibliographystyle{plainnat} 
    \bibliography{References}
\end{document}






